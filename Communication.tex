\documentclass{article}
\author{Matteo Secco}
\title{Communication and Argumentation}

\usepackage{hyperref, titlesec}
\newcommand{\sectionbreak}{\clearpage}

\usepackage{mathtools}

\begin{document}
\maketitle\newpage
\tableofcontents\newpage

\section{Introduction}
\paragraph{Course themes}
\begin{itemize}
\item Linguistic
\item Retoric
\item Presentation Skills
\item Argumentation (and Manipulation)
\item Grant-writing
\item How to write papers
\end{itemize}

\paragraph{How to learn communication (according to Cicero)}
\begin{itemize}
\item Study the theory
\item Copy the technique of good communicators
\item Prective on communication
\end{itemize}

\paragraph{How we'll learn during the course}
\begin{itemize}
\item Classes
\item Labs
\item MOOC (like coursera)
\item Tutoring for the final project
\end{itemize}

\paragraph{Exam structure}
60\% of the grade from written test, 40\% from project.
\subparagraph{Written test} will consist of 6 open questions. One of those \textbf{will be} "Find 4 examples of arguments in your field of study, identify and discuss them"
\subparagraph{Project} simulation of a proposal for PoliHub/Research Council




\section{Communication: basic principles}
Languages are made of signs. The existence of the concept implies the existence of an intelligence capable of relating two elements: \textbf{signifier} and \textbf{signified}. For example, fire by itself is just smoke, but intelligence can read it as a sign of a fire.
\begin{description}
\item[Signifier] is the element used as a sign: the smoke in the example above
\item[Signified] is the element the sign stands for: the fire in the example above
\end{description}


\subsection{Four big consequences of using signs}
\paragraph{Allow to transmit knowledge beyond direct aquaintance} Of course we acquire some of our knowledge by direct experience. Symbolical representation of languages allow knowwledge to be transmitted even in absence of direct acquaintance.\\ 
This can be achieved because signs work by abstraction: from a concrete object we abstract the concept of it, and are able to use it as a sign.\\
Anyway, without direct acquaintance, no \underline{new} knowledge can be generated
\paragraph{Allows to expand knowledge by  combining concepts toghether} Languages are like LEGO blocks: we can combine signs to obtain new ones. For example, \textit{Not+Finite=Infinite}. The concept of infinite cannot be experienced ever, but we are able to get it combining other concepts.
\paragraph{We can create thinghs that does not exist} For example we can create the concept of \textit{Teleportation=$\overbrace{Tele}^{distance}$+$\overbrace{Port}^{moving}$}
\paragraph{We can lie}


\subsection{How signs work}
\label{subsec:singAndAmbiguity}
\paragraph{What is a sign?} It is composed by a signifier and a signified
\begin{description}
\item[Signifier] is the element bringing a meaning
\item[Signified] the meaning
\end{description}
Signs are, however, arbitrary. Different types of arbitraries exist:
\begin{description}
\item[Arbitrary of the signifier:] If we all agree to call books "Kikoo", language would still work
\item[Arbitrary of the signified:] The same sign (word) can represent different object: the italian word "porta" can be translated to "door" or to "brings"
\end{description}

\subsection{What is a "text"?}
\paragraph{Text} is an extended structure of syntactic units, any meaningful linguistic production, both oral and written.\\
\paragraph{Text interpretation} is mostly non verbal, as the verbal components are the little part.
\paragraph{Co-text}  is essential to understand the overall meaning. Co-text refers to the "text around". It is determined by both:
\begin{description}
\item[the other sentences:] Aldo asked me 10€. $\substack{\text{He loaned me 5€}\\\text{He loaned me 15€}}$. $\rightarrow$ The picked sentence determines if Aldo is kind or greedy.
\item[the arrangement of other sentences:] Louis is good. His father bought him a pc $\rightarrow$ pc is a reward\\
His father bought him a pc. Louis is good $\rightarrow$ good at using the pc
\end{description}
\paragraph{Con-text} Is the world around the text. The context contains the cotext, but also includes elements of knowledge that are external from the text itself.
\paragraph{Encyclopedia} represents our knowlegde of the world: without external knowledge wwe would not get any conceptual difference between green eyes and red eyes. But from our personal encyclopedia, we know that red eyes are uncommon, and are maybe a sign for a disease, while green eyes are pretty normal.

\subsection{Elements of good communication}
\paragraph{Reference} Communication must be understandable: a good communicator must be able to identify a common ground for communication (common knowledge).
\paragraph{Common ground mistakes}
\begin{itemize}
\item Taking for granted \underline{not shared} knowledge
\item Explaining knowledge that is \underline{already known}
\end{itemize}




\section{Rethoric}
Rethoric ("the art of speaking") is born in the greek democracy.\\
It is an art, with principles and rules.\\
By itself, rethoric is neutral, but it can be used to both manipulate or to persuade.\\
\paragraph{Bad use of rethoric} In greek there was a rethor, Corax, who instructed another man, Tisias, on rethoric.\\
Tisias refused to pay Corax, and in court he defended itself saying "If i loose the trail, I don't have to pay Corax, as he was not a good teacher. If I win the case, I don't have to pay Corax by law".
\paragraph{Elements of  rethorics:} We will divide rethoric into substeps:
\begin{itemize}
\item Finding good arguments
\item Arranging arguments propterly
\item Choose the best style for the speech
\item Put the audience at the center of our attention
\item Calibrate our communication based on the common ground
\end{itemize}


\subsection{Invention: finding what to say}
Invention is finding what to say o create our communication. Invantion is brainstorming.\\
Ancient orators suggest to take into consideration three aspects:
\begin{description}
\item[ethos:] the credibility of the speaker
\item[pathos:] the effects on the audience
\item[logos:] the facts about the things under discussion
\end{description}
For example, in a typical project proposal people focuses a lot on thechnical facts (logos). A level up is to point out how you will be capable of running the project (ethos) and the benefits the project will bring to the company (pathos)
\paragraph{Prepare to objections} anticipating objections allows us to reply brilliantly, avoiding to freeze and create an embarassing situation.\\
It is also useful, for our own organization, to categorize the possible objection, in order to be more efficient in preparing the answers and in remembering it
\paragraph{Anticipate wishes} What is the audience wishing? Anticipating the wishes allows us to create a sensation of satisfaction, which will lead to a positive overall perception
\paragraph{Brain-mapping} After we think we gathered enought material, we want to group it. A possible grouping may be ethos-pathos-logos, but also more domain-specific groupings are possible\\
Grouping allows us to check weather we have enough of everything.
\paragraph{Conclusion} Invention is crucial. Being essentially brainstorming, it works better if performed in group. Also, keep in mind that invention will always be "opened", it will not end after the first time you perform it. The contents will change overtime, influenced by new information. Some new topics may emerge, some other may become not relevant.\\
Finally, keep in mind that invention takes time. Ideas come where they want, not when you command





\appendix
\section{Communication Tips}
\paragraph{Acronyms} must always be explained at least once.
\paragraph{Correctness} of grammar and semantic.
\paragraph{Be informative} Saying "he has 2 eyes, 1 mounth" is useless when describing a person.
\paragraph{Be relevant} ensure to answer any open question, be it explicit or implicit.
\section{Likely exam questions/arguments}
\paragraph{Find 4 example of arguments in your fiels, identify and discuss them}
\paragraph{Ambiguity of signs} \ref{subsec:singAndAmbiguity}
\end{document}