\documentclass{article}
\author{Matteo Secco}
\title{Communication and Argumentation}

\usepackage{hyperref, titlesec}
\newcommand{\sectionbreak}{\clearpage}

\usepackage{mathtools}

\begin{document}
\maketitle\newpage
\tableofcontents\newpage

\section{Introduction}
\paragraph{Course themes}
\begin{itemize}
\item Linguistic
\item Retoric
\item Presentation Skills
\item Argumentation (and Manipulation)
\item Grant-writing
\item How to write papers
\end{itemize}

\paragraph{How to learn communication (according to Cicero)}
\begin{itemize}
\item Study the theory
\item Copy the technique of good communicators
\item Prective on communication
\end{itemize}

\paragraph{How we'll learn during the course}
\begin{itemize}
\item Classes
\item Labs
\item MOOC (like coursera)
\item Tutoring for the final project
\end{itemize}

\paragraph{Exam structure}
60\% of the grade from written test, 40\% from project.
\subparagraph{Written test} will consist of 6 open questions. One of those \textbf{will be} "Find 4 examples of arguments in your field of study, identify and discuss them"
\subparagraph{Project} simulation of a proposal for PoliHub/Research Council




\section{Communication: basic principles}
Languages are made of signs. The existence of the concept implies the existence of an intelligence capable of relating two elements: \textbf{signifier} and \textbf{signified}. For example, fire by itself is just smoke, but intelligence can read it as a sign of a fire.
\begin{description}
\item[Signifier] is the element used as a sign: the smoke in the example above
\item[Signified] is the element the sign stands for: the fire in the example above
\end{description}


\subsection{Four big consequences of using signs}
\paragraph{Allow to transmit knowledge beyond direct aquaintance} Of course we acquire some of our knowledge by direct experience. Symbolical representation of languages allow knowwledge to be transmitted even in absence of direct acquaintance.\\ 
This can be achieved because signs work by abstraction: from a concrete object we abstract the concept of it, and are able to use it as a sign.\\
Anyway, without direct acquaintance, no \underline{new} knowledge can be generated
\paragraph{Allows to expand knowledge by  combining concepts together} Languages are like LEGO blocks: we can combine signs to obtain new ones. For example, \textit{Not+Finite=Infinite}. The concept of infinite cannot be experienced ever, but we are able to get it combining other concepts.
\paragraph{We can create things that does not exist} For example we can create the concept of \textit{Teleportation=$\overbrace{Tele}^{distance}$+$\overbrace{Port}^{moving}$}
\paragraph{We can lie}


\subsection{How signs work}
\label{subsec:singAndAmbiguity}
\paragraph{What is a sign?} It is composed by a signifier and a signified
\begin{description}
\item[Signifier] is the element bringing a meaning
\item[Signified] the meaning
\end{description}
Signs are, however, arbitrary. Different types of arbitraries exist:
\begin{description}
\item[Arbitrary of the signifier:] If we all agree to call books "Kikoo", language would still work
\item[Arbitrary of the signified:] The same sign (word) can represent different object: the Italian word \textit{porta} can be translated to \textit{door} or to \textit{brings}
\end{description}

\subsection{What is a "text"?}
\paragraph{Text} is an extended structure of syntactic units, any meaningful linguistic production, both oral and written.\\
\paragraph{Text interpretation} is mostly non verbal, as the verbal components are the little part.
\paragraph{Co-text}  is essential to understand the overall meaning. Co-text refers to the "text around". It is determined by both:
\begin{description}
\item[the other sentences:] Aldo asked me 10€. $\substack{\text{He loaned me 5€}\\\text{He loaned me 15€}}$. $\rightarrow$ The picked sentence determines if Aldo is kind or greedy.
\item[the arrangement of other sentences:] Louis is good. His father bought him a pc $\rightarrow$ pc is a reward\\
His father bought him a pc. Louis is good $\rightarrow$ good at using the pc
\end{description}
\paragraph{Con-text} Is the world around the text. The context contains the cotext, but also includes elements of knowledge that are external from the text itself.
\paragraph{Encyclopedia} represents our knowlegde of the world: without external knowledge wwe would not get any conceptual difference between green eyes and red eyes. But from our personal encyclopedia, we know that red eyes are uncommon, and are maybe a sign for a disease, while green eyes are pretty normal.

\subsection{Elements of good communication}
\paragraph{Reference} Communication must be understandable: a good communicator must be able to identify a common ground for communication (common knowledge).




\section{Rhetoric}
Rhetoric ("the art of speaking") is born in the Greek democracy.\\
It is an art, with principles and rules.\\
By itself, rhetoric is neutral, but it can be used to both manipulate or to persuade.\\
\paragraph{Bad use of rhetoric} In Greek there was a rhetor, Corax, who instructed another man, Tisias, on rhetoric.\\
Tisias refused to pay Corax, and in court he defended itself saying \textit{If i loose the trail, I don't have to pay Corax, as he was not a good teacher. If I win the case, I don't have to pay Corax by law}.
\paragraph{Elements of  rhetorics:} We will divide rhetoric into sub-steps:
\begin{itemize}
\item Finding good arguments
\item Arranging arguments properly
\item Choose the best style for the speech
\item Put the audience at the center of our attention
\item Calibrate our communication based on the common ground
\end{itemize}


\subsection{Invention: finding what to say}
Invention is finding what to say o create our communication. Invention is brainstorming.\\
Ancient orators suggest to take into consideration three aspects:
\begin{description}
\item[ethos:] the credibility of the speaker
\item[pathos:] the effects on the audience
\item[logos:] the facts about the things under discussion
\end{description}
For example, in a typical project proposal people focuses a lot on technical facts (logos). A level up is to point out how you will be capable of running the project (ethos) and the benefits the project will bring to the company (pathos)
\paragraph{Prepare to objections} anticipating objections allows us to reply brilliantly, avoiding to freeze and create an embarrassing situation.\\
It is also useful, for our own organization, to categorize the possible objection, in order to be more efficient in preparing the answers and in remembering it
\paragraph{Anticipate wishes} What is the audience wishing? Anticipating the wishes allows us to create a sensation of satisfaction, which will lead to a positive overall perception
\paragraph{Brain-mapping} After we think we gathered enough material, we want to group it. A possible grouping may be ethos-pathos-logos, but also more domain-specific groupings are possible\\
Grouping allows us to check weather we have enough of everything.
\paragraph{Conclusion} Invention is crucial. Being essentially brainstorming, it works better if performed in group. Also, keep in mind that invention will always be "opened", it will not end after the first time you perform it. The contents will change overtime, influenced by new information. Some new topics may emerge, some other may become not relevant.\\
Finally, keep in mind that invention takes time. Ideas come where they want, not when you command

\subsection{Arrangement: organizing the communication} Is the act of picking the best order of our argument. The ways to arrange depends on the context: for example whether the communication will be oral or written, may be interrupted or not, and other
\paragraph{Nestorian order} can be used in both written and oral communication. It requires to have the full and continuous attention of the audience $\rightarrow$ no interruptions. This is because if the communication is interrupted, strong arguments may be lost.\\
The Nestorian structure is $\textbf{Strong arguments}\rightarrow\textbf{Weak arguments}\rightarrow\textbf{Strong arguments}$.
\paragraph{Descending climax} The structure is $\textbf{Strongest arguments}\rightarrow\textbf{Strong arguments}\rightarrow\textbf{Weak arguments}\rightarrow\textbf{Weakest arguments}$.
\paragraph{Ascending climax} The structure is $\textbf{Weakest arguments}\rightarrow\textbf{Weak arguments}\rightarrow\textbf{Strong arguments}\rightarrow\textbf{Strongest arguments}$. This is desirable when we want to be sure to have an emotional effect, and we are sure that we will not be interrupted.

\subsubsection{Beginning of the speech}
\paragraph{Captatio benevolentiae} is the act of winning the favor of the audience. It can be achieved in three ways:
\begin{itemize}
\item Pleasing the audience
\item Putting yourself into the shoes of the audience
\item Playing humble and be self ironic
\end{itemize}


\subsection{Style: dressing ideas in words}
There are more ways in which you can express an idea, and the way we choose will make a difference on how the content is perceived. In this section we will learn how to choose one.
\paragraph{Elocutio} reminds that  a good text has to be correct, clear and appropriate to the situation
\paragraph{Ornament} is using  rhetorical figures to deflect the normal use of language with the purpose of provide a vivid image
\subsubsection{Rhetorical figures}
\paragraph{Metaphor} making reference to something by means of something else.
\paragraph{Analogy} a comparison meant to help the audience to connect to an idea.
\paragraph{Anaphora} is the repetition of the same word/expression at the beginning of successive clauses.
\paragraph{}\textit{Suggested to dig out for more by ourself}


\subsection{Common Ground}
\textit{The collection of mutual knowledge, mutual beliefs and mutual assumptions that is essential for communication between two people} (Herbert Clark)\\
We will consider, in general, Common Ground as Shared Knowledge.
\paragraph{Common ground mistakes}
\begin{itemize}
\item Taking for granted \underline{not shared} knowledge
\item Explaining knowledge that is \underline{already known}
\end{itemize}

\subsection{Proposed exercises}
\paragraph{Exercise 1: "good plagiarism"}
Don’t be afraid, it’s ok. And do be sure that I know that “real” plagiarism is bad. But it can also be used for the good, as in this exercise.\\
Cicero said that you learn by imitation. You find a report that according to you is well organized? Try to “copy” it, organizing YOUR report using that schema. You find an “introduction” paragraph that arranges arguments in a way you feel is effective?\\
Try to do the same for the introductory paragraph of YOUR scientific paper. And so on.\\
\paragraph{Exercise 2: “Talk to the most unlikely audience”}
This exercise is about common ground and being clear.\\
Take a technical topic (related to your research) and make a presentation (e.g. a 15 minutes oral presentation) meant for somebody with a different background (e.g. in humanities) or with a different education level (e.g. high-school students or even primary school children) or for someone completely far from the field you are active in (e.g. your grand-father?).\\
Try to find a way to actually deliver the presentation and get feedback, it will be eye-opening!\\
\paragraph{Exercise 3: “Give yourself a goal”}
This exercise is about invention. The same topic (e.g. a research activity) can give vent to quite different kinds of communication, according to a number of parameters (the goal, the audience, the context…). In this exercise, you are required to work on the goal.\\
Create a presentation about the topic of your research with the goal of getting funds for it, another with the goal of conveying to fellow scientists the significance of your work, etc.\\
If you have the option to actually find someone who might fit into your hypothetical scenario, give the presentation and collect feedback. But even if you can’t, the sheer exercise of choosing the proper topics (exercise on “key messages”) to fit the goal will bring substantial benefits to your communication skills.

\subsection{Method, Plan, Resources, Proponents}
\paragraph{Method} How much can you predict the future? Two extremes:
\begin{itemize}
\item Provide a very detailed plan, like month-by-month. Generally unsuitable for research projects
\item Be vague, discusse the goals but be fuzzy on deadlines. But it may look vague and untrustable.
\end{itemize}
Credibility has a dramatic role: the more credible you are (or your institution, or references) the vaguer you can be.
\paragraph{Temporal plan} Better approach is to reference to the given time: what can you do in the (for example) 12 mounths given? Can you delier a final product? A design? A prototype?
\paragraph{Resources} How much will the project cost to the reviewer? 
\begin{itemize}
\item How can you fit the project in the available budget?
\item Pay attention to the "value for money" issue: if you ask too much you are greedy, too few and you're not credible
\item Have a budget consistent with the plan, in particular regarding human resources
\end{itemize}
\paragraph{Proponent}
\begin{itemize}
\item Introduce yourself as someone credible
\item Mention your achievent, past projects, prizes, even if not related to the project. 
\end{itemize}


\section{How to persuade}
\subsection{Introduction}
\paragraph{Argument} action of supporting different opinions
\paragraph{Argumentum ad populum} Fallacious argument that the most popular opinion is the most correct
\paragraph{Usages of argumentation in science}
\begin{itemize}
	\item Defend the relevance of your research topic
	\item Persuade investors/partners
	\item Get accepted for an internship
\end{itemize}
\subsection{Syllogisms}
\paragraph{Syllogism} Form of reasoning structured as follows:\\
Major premise + Minor premise $\implies$ Conclusion\\
In syllogisms, the major premise is a \underline{universal truth}
\paragraph{Enthymemes} Form of reasoning structured as syllogisms, but where the major premise is not universally true but instead \underline{generally} true. Enthymemes are the base of discussions.
\subsection{Ways to arguments} 
\paragraph{Facts and data} do not need to be shaped: they just need to be found, collected, selected, and placed at the right moment.
\paragraph{How to respond to data}
\begin{description}
	\item[Dispute the fact] \textit{This is false}
	\item[Challenge the relevance] \textit{This is not relevant to the point we're making}
	\item[Dispute that the data is partial] and so not valid
	\item[Outnumber the data with more other data]
\end{description}
\paragraph{Topoi} patterns of reasoning stored in our cognitive functionalities
\subparagraph{The more and less likely} "If the most likely thing does not happen, then the less likely thing also will not happen
\subparagraph{Contraries} Based on the principles of not contradiction: "If A is true, then $\neg A$ is false
\paragraph{Examples} trigger inductive form of reasoning. 
\subparagraph{Similarity} By showing a similar argument, trigger a generalization "moving the truth" of the similar into the original
\subsection{Quasi-logic arguments}
Derive their strength from the resemblance to mathematical/logical demonstrations
\paragraph{Reciprocity} argument relying on the assumption of symmetry between two situations $\implies$ what applies to one must apply to the other
\subparagraph{Objecting} can be performed only by negating the validity of the symmetry
\paragraph{Rule of justice} \textit{Two beings or situations falling under the same category or genre should be dealt with in the same manner}
\paragraph{Complex question} taking for granted something that might deserved to be discussed
\paragraph{Transitivity} $A:B=B:C$ \textit{our friend's friends are our friends}
\subparagraph{Objecting} invalidate the parallelism
\paragraph{Sacrifice} states that if something costs a lot it must be valuable
\paragraph{Quantity} \textit{A lot means good}: a lot of publications means a scholar is good

\subsection{Fallacies}
Fallacies are arguments that are apparently correct in form but invalid
\paragraph{Appeal to ignorance} If a proposition has not been disproved, then it cannot be considered true, then it must be considered true. \textit{God exists because nobody disproved its existence}
\paragraph{Straw man fallacy} modify the interlocutor thesis from A to B, and then objecting to B.
\paragraph{Irreparable direction} Draw something as a \textit{now or never} choice, implying that it is good (works because people fear regret)
\paragraph{Argument from authority} Appealing to an authority (not necessarily a juridical one, an important person in the field is ok too) to support the thesis. Authorities are not truth by the way.
\subsection{Proposed exercises} 
\paragraph{Exercise 1: “Mary dyed her hair green”}
Take the “Mary dyed her hair green” dialogue and try to identify the arguments used in it and the “hidden assumptions” the enthymemes at work in it are based on. Then, try to come up with further ways to reply to those arguments.\\
Decide which side you want to support and fight for that! You may want to share in the course’s forum, to get your peers’ feedback.
\paragraph{Exercise 2: "The Titanic dilemma”}
Imagine you are safely on board one of the Titanic’s life-boats, watching as the ship sinks. You are discussing with the other people on the boat: shall we go back and try to save more lives or shall we keep “safe” where we are? Brainstorm about arguments in favor of both stands (in turn: 15 minutes in favor of one, 15 minutes in favor of the other).\\
If you can find someone willing to do this exercise with you, take one position and ask the other person to play the “opponent”. This a small debate exercise, focused on invention.
\paragraph{Exercise 3: “It is as if”}
Take a claim you would like to support and try to find arguments by analogy (the “it is as if” way).
\paragraph{Exercise 4: “Arguments’ reverse engineering”}
Take a scientific paper and try to identify if there is any “argument” embedded in it (by analogy, of the more or less likely, of contrary, of authority…).
\paragraph{Exercise 5: “The going backward dialogue”}
In order to train yourself to identify the major premises of enthymemes, you can perform the “going backward dialogue”. Number one: select a victim (a good friend of yours, your mother…). Start a dialogue on whatever topic you want (“How can we save humanity?” or “What should we eat for dinner?”).\\
Instead of moving on in the dialogue, identify the major premises of what the interlocutor says and put them into question, thus “going backward” instead of cooperating (“moving forward”). This is quite annoying for the interlocutor so I strongly suggest to stop after a while and apologize. Apart from training yourself into the identification of hidden premises (which is always useful), you will come up with interesting discoveries on what people base their everyday reasonings on (I call it their “metaphysics”)…\\
Discuss your findings in the forum.



\section{Writing a project proposal}
\paragraph{Proposal types}
\begin{description}
\item[Market-oriented proposal] for projects having the purpose to create a sellable product. You want to convince that the product can be developed, sold, has a market share, is concurrent with the competitors, and ultimately will allow the company to make money.
\item[Research proposal] for R\&D. You want to appear as visionary, creative and unique. To convince that you are actually advancing the knowledge of the sector.
\end{description}

\paragraph{Proposal "mandatory" topics/contents}
\begin{description}
\item[Opportunity] Which good conditions make the project a good pick?
\item[Goals and objectives] What do we want to achieve with the project?
\item[State of the Art] The situation in which the project is born
\end{description}

\paragraph{Measures to include}
\begin{description}
\item[Method] how will the project be carried out? Is it a well-known and effective method?
\item[Resources] Budget, materials...
\item[Temporal plan] How long will it take? Maybe also include a Gannt, including mid-work deliverable schedule
\item[Proponents] Show that your team is able to actually complete the project
\end{description}

\subsection{Diffusion theories}
\label{subsec:diffusion_theories}
\paragraph{Diffusion theory} a theory trying to explain how innovations impact society. Useful in project proposals to better understand the long-term effects of your project\\
An innovation can be seen as the introduciton of a new species in an ecosystem. 
\begin{itemize}
\item It can destroy much existing things
\item It can be destroyed
\item It can cohexist well with existing things
\end{itemize}


\subsection{Executive sumary}
\paragraph{Executive summary} Like the abstract in a paper. Short description to convince the reader to read further.
\paragraph{Typical reading scheme}
\begin{itemize}
\item Description of the project
\item Financial part
\item Who are the proponents
\end{itemize}
Anyway, linear reading is quite rare. So the proposal must be understandable no matter from where it is started. \underline{Don't fear to repeat the key concepts in different sections}

\paragraph{Writing an executive summary}
\begin{description}
\item[Stay short] don't exceed 10\% of the total length of the proposal
\item[Respect editorial format] For example, don'e exceed a page size by just a few lines: better shrink it
\item[Provide a general framework of the proposal] It must frame the proposal, but \underline{not being just an introduction}. It must be a \underline{synthesis}.
\item[Must entail the essential elements of the proposal]: Opportunity, methos, goals, state of the art
\item[Present the best of the proposal] Including strong points, interesting data, cruciality of solving the issue, extraordinarity of the benefits
\item[Go straight to the point] without consuming too much time, you're addressing a busy reader.
\end{description}

\subsection{Starting point, Goal, State of the Art.} 
These elements will be found in any proposal on earth.
\paragraph{Starting point} 
The proposal can start from \underline{an issue} you aim to solve. You may want to let the reader feel like the person that can save the world\\
Also, you can start by identifying \underline{a gap}, like some hole in the knowledge on some topic. Here again, a "call of duty" approach works well for convincing the audience.\\
\underline{Opportunity} also works well. "You can gain a lot of money" always is appealing.
\paragraph{Goal} The goal of the goal is to prove why your project is relevant for different aspects: to your company, to your operative sector, to the industry, and to society. A project sjould have profound reasons to be carries out, having effects as wide as possible.
\paragraph{State of the Art} We want to prove that we are connected with the world, that we are not reinventing the wheel, that we can spot similar works, to take advantage of them, and that we know how our work will insert in the world. See \ref{subsec:diffusion_theories}





\section{Scietific writing}
The audience is of specialists, the purpose is to share our results and eventually allow repeatibility. The main kind of scientific contribution are:
\begin{description}
\item[Original research] or primary literature. Deals with experiments in which data is gathered and analyzed.
\item[Literature reviews] or secondary literatures. Consists in critical scanning of existing literature. Quoted much often.
\item[Case-studies descriptions] Qualitative description and analysis of something that cannot be generalized.
\end{description}
\paragraph{Hero-journey macro schema}
\begin{itemize}
\item We meet the hero in its normal life
\item Something does not work in its life
\item It is clear that the hero will soon be called to action
\item An "accidental" opportunity takes place, moving the hero to action
\item The journey begins, with pitfalls 
\item The hero learns soething changing everything
\item The initial situation is restored and improved
\end{itemize}

\subsection{Main sections \& abstract}
\paragraph{Abstract} The marrow of the work. ust be self-contained, can be thought as an independent document. It is allowed to copy-paste from the main doument!\\
The abstract is usually a quite short text. A standart schema could be:
\begin{enumerate}
\item What the paper is about
\item The goal of the work and its overall significance
\item [The method used], skippable
\item Main results
\item Discussion/conclusions
\end{enumerate}
Keep in mind: the abstract is not an introduction, is not an index. It has to convey a message by itself!
\paragraph{Introduction} Which questions are you dealing with? Why is it relevant? How will your research help in answering them? What is the context?\\
The introduction can also contain an index of the paper.
\paragraph{Method} The main goal of the method section is to explain how the study was taken. You want to provide all the elements to replicate your study. A common issue is the tendency to \underline{take for granted} steps you did. Make someone else read this section!
\paragraph{Results, conclusions} Results convey the empirical results, data. Is is good to use graph or tables. Underline which data has been eliminated.\\ 
Conclusion is the "human" analysis of the nude numbers. Draw lessons out of your results (which are in line/related to) the goals of the paper. Goals can be referenced (even by repeating them). Also some unexpected discoveries should be included, but it should be clear that it was unexpected and why it still is interesting. The conclusion also should contain references to other papers, and limitations of the study.
\paragraph{Aknowledgement} to thank the publisher, colleagues, volounteers.



\appendix
\section{Communication Tips}
\paragraph{Acronyms} must always be explained at least once.
\paragraph{Correctness} of grammar and semantic.
\paragraph{Be informative} Saying \textit{he has 2 eyes, 1 mouth} is useless when describing a person.
\paragraph{Be relevant} ensure to answer any open question, be it explicit or implicit.

\section{Argumentation tips}
\paragraph{Choose good major premises} in particular, choose ones that the interlocutor agrees on
\paragraph{Identify the major premises of the interlocutor} to decide if you agree, and in case doubt about the major premise instead of less relevant elements
\paragraph{Leave the conclusion implicit} may bring satisfaction to the interlocutor, leading to a more solid adhesion to the thesis. This should not be done for key points.
\paragraph{Take some arguments for granted} reduces the chance of them to be discussed

\section{Project proposals}
\paragraph{Rewrite the executive summary} again and again and again. It must be super-effective
\paragraph{Write the executive summary after having written the proposal}
\paragraph{Write your own review in the proposal}
\paragraph{Know the evaluation criteria}
\paragraph{Repeat the keywords of the project call}
\paragraph{Highlight sentences answering the reviewer's questions.} Bold them, box, them. It's not unprofessional, it's smart

\section{Likely exam questions/arguments}
\paragraph{Find 4 example of arguments in your field, identify and discuss them}
\paragraph{Ambiguity of signs} \ref{subsec:singAndAmbiguity}
\end{document}